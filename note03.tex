\documentclass{notes}
\begin{document}
\entetenote{Paul Lévy, Fermat-Wiles et une somme.}{3}{}

Dans \cite{Levy}, Paul Lévy affirme (sans la démontrer) l'équivalence entre l'identité suivante
\[ \sum_{p=3}^{+\infty} \frac{1}{p^{2}} \int_{0}^{2\pi} \left( \sum_{n=1}^{+\infty} \frac{\cos\left( n^{p }x \right)}{n^{2}} \right)^{3}  \d x = 0  \]
et le théorème de Fermat-Wiles (seulement Fermat à l'époque). On détaille le sens direct ici.

Montrons en fait que pour tout entier $p\geq 3$, 
\[ \int_{0}^{2\pi} \left( \sum_{n=1}^{+\infty} \frac{\cos\left( n^{p }x \right)}{n^{2}} \right)^{3}  \d x = 0 .  \]
Fixons $p$ un tel entier. Commençons déjà par remarquer que la série de fonctions sous l'intégrale converge normalement -- et donc son cube, uniformément -- sur $[0,2\pi]$. Ainsi il suffit de vérifier que pour tout entier $N\geq 1$ on a 
\[  \int_{0}^{2\pi}   \left( \sum_{n=1}^{N} \cos(n^{p}x) \right)^{3} \d x = 0. \]
Pour cela, il est suffisant de montrer que si $n_1,n_2,n_3$ désignent des entiers supérieurs ou égaux à $1$ alors 
\[ \int_{0}^{2\pi} \cos(n_1^{p} x)\cos(n_2^{p}x) \cos(n_3^{p}x) \d x = 0.\]

L'identité pour tout réels $r,s,t$ 
\[\cos \left( r \right) \cos \left( s \right) \cos \left( t \right) = \frac 14 \left( \cos(r+s-t)+\cos(s+t-r) + \cos(t+r-s) + \cos(r+s+t) \right)\]
%permet de 
%primitiver par rapport à $x$ la quantité $ \cos(ax)\cos(bx)\cos(cx) $  en 
%\[ \frac 14 \left[    \frac{\sin \left( (a+b-c)x \right)  }{a+b-c} + \frac{\sin\left( (b+c-a)x \right)}{b+c-a}  + \frac{\sin\left( (c+a-b) x\right)  }{c+a-b} + \frac{\sin\left( (a+b+c)x \right)}{a+b+c}    \right] \]
%où l'on utilise le prolongement par continuité en $0$ de sinus cardinal,
permet de voir que la fonction intégrée ci-dessus est bien de moyenne nulle sur $[0,2\pi]$ avec $a=n_1^{p},\ b=n_2^{p},\ c=n_3^{p}$, dès lors que $a+b-c \neq 0,\ b+c-a \neq 0,\ c+a-b \neq 0$ ce qui est vrai par le théorème de Fermat-Wiles car $p \geq 3$.  


\begin{rem}
  Lors de la publication de \cite{Levy}, le théorème de Fermat-Wiles n'était encore que le grand théorème de Fermat et pas encore démontré. En fait cette identité est une façon de compter les solutions à l'équation de Fermat : remarquons que si $(n_1,n_2,n_3)$ en est une solution alors alors elle apporte une contribution $+1/4$ dans l'intégrale ci-dessus (au total dans l'identité il faudrait aussi prendre en compte les autres termes développés du cube qui tombent sur $n_1,\ n_2$ et $n_3$).  Ainsi, une façon de prouver le théorème de Fermat-Wiles serait de réussir à démontrer cette identité, mais ceci semble hors de portée. Pour aller plus loin, on pourra se renseigner sur la méthode du cercle de Hardy et Littlewood \cite{HW}.   Je remercie enfin Antoine Chambert-Loir qui m'a montré cet article de Paul Lévy.
\end{rem}




\bibliographystyle{siam}
\bibliography{refsnote} 



\end{document}
