\documentclass{notes}
\begin{document}
\entetenote{Calcul de $\int_0^{+\infty} \left( \frac{\sin(x)}{x} \right)^{n}\d x$ via la loi de Bates}{2}{}

Il est bien connu que $\int_{0}^{+\infty} \frac{\sin(x)}{x} \d x = \frac{\pi}{2}$ est une intégrale semi-convergente au sens de Riemann. Montrons que pour tout $n\in\N^{*}$ on a 
\[ \ds\int_0^{+\infty} \left( \frac{\sin(x)}{x} \right)^{n}\d x = \frac{\pi}{4(n-1)!}  \sum_{k=0}^{n} (-1)^{k}\binom nk \left( \frac n2 - k \right)^{n-1}\mathrm{sgn}\left( \frac n2 - k \right)\] 

\begin{proof}
  Nous allons utiliser la transformée de Fourier des fonctions $\mathrm L^{2}$ avec la convention suivante : 
  \[ \hat f(\xi) = \int_{-\infty}^{+\infty} f(x) e^{-ix \xi} \d x. \]

  Soit $n\geq 1$. Notons $f\colon x \mapsto \frac{\sin(x)}{x}$. Alors $f\in\mathrm L^{2}(\R)$ et il est bien connu qu'avec notre convention, $\hat f = \pi \mathbf 1_{[-1,1]}$. Remarquons maintenant que par parité, 
  \begin{equation}
    \label{in1}
    I_n \defeq \int_{0}^{+\infty} \left( \frac{\sin(x)}{x} \right)^{n} \d x = \frac 12 \int_{-\infty}^{+\infty} \left( \frac{\sin(x)}{x} \right)^{n} \d x = \frac 12 \widehat{f^{n}}(0). 
  \end{equation}
  Or avec notre convention on a pour tout $(f,g)\in (\mathrm L^{2})^{2}$, $\widehat{fg}= \frac 1{2\pi} \hat f \ast \hat g$ (où $\ast$ désigne le produit de convolution), si bien que \eqref{in1} devient  
  \begin{equation}
    \label{in2}
    I_n = \frac 12 \frac{1}{(2\pi)^{n-1}} \left( \hat f\ast \cdots \ast \hat  f \right)(0).
  \end{equation}

  Faisons un petit détour par les probabilités. Prenons $X_1,\dots,X_n$ des variables indépendantes uniformes sur $[-1,1]$. La loi de $S_n \defeq \frac{X_1+\cdots+X_n}{n}$ est alors connue, c'est une loi de \emph{Bates} (voire par exemple \cite{JKB} pour la densité sur $[0,1]$ puis faire une transformation affine pour passer sur $[-1,1]$)) qui a pour densité 
  \[ g_{n} \defeq  \frac 12 \frac{n}{2(n-1)!}  \mathbf 1_{[-1,1]} \sum_{k=0}^{n} (-1)^{k} \binom nk \left( n\frac{x+1}{2}- k \right)^{n-1} \mathrm{sgn}\left( n\frac{x+1}{2}- k \right) \]
  où $\mathrm{sgn}$ signifie \emph{signe}, avec la convention $\text{sgn}(0)=0$. Remarquons que $\hat f = \underbrace{\frac 12 \mathbf 1_{[-1,1]}}_{\defeq \hat g} 2\pi$ où $\hat g$ est la densité des $X_i$. Ainsi $\hat g \ast \cdots \ast \hat g$ qui est la densité de $nS_n$ vérifie (effet d'une transformation affine sur la densité) : 
  \[ \left( \hat g \ast \cdots \ast \hat g \right)(x) = \frac 1n g_{n}\left( \frac xn \right) =  \frac 1{4(n-1)!} \mathbf 1_{[-1,1]}(x) \sum_{k=0}^{n} (-1)^{k} \binom nk \left( \frac{x+n}{2}-k \right)^{n-1} \mathrm{sgn} \left( \frac{x+n}{2}-k  \right) \] 
  donc \[ \left( \hat g\ast \cdots \ast \hat g \right)(0) = \frac 1{4(n-1)!} \sum_{k=0}^{n} (-1)^{k}\binom nk \left( \frac n2 - k \right)^{n-1}\mathrm{sgn} \left( \frac n2 - k \right). \]

  Enfin, par \eqref{in2} comme $\hat f = 2\pi \hat g$ on récupère 
  \[ I_n = \frac 12 \frac{(2\pi)^{n}}{(2\pi)^{n-1}} \left( \hat g\ast \cdots \ast \hat  g \right)(0)  =  \frac{\pi}{4(n-1)!}  \sum_{k=0}^{n} (-1)^{k}\binom nk \left( \frac n2 - k \right)^{n-1}\mathrm{sgn}\left( \frac n2 - k \right) .  \]


\end{proof}

Ci-dessous les valeurs de $\frac{I_n}{\pi}$ sous forme de fractions irréductibles pour $n\in \ent 1{30}$, calculées grâce au module \verb+fractions+ de Python.

\begin{lstlisting}
1/2
1/2
3/8
1/3
115/384
11/40
5887/23040
151/630
259723/1146880
15619/72576
381773117/1857945600
655177/3326400
20646903199/108999475200
27085381/148262400
467168310097/2645053931520
2330931341/13621608000
75920439315929441/457065319366656000
12157712239/75277762560
5278968781483042969/33566877054287216640
37307713155613/243290200817664
9093099984535515162569/60740063241091153920000
339781108897078469/2322315553259520000
168702835448329388944396777/1178600187130132750663680000
75489558096433522049/538583682060103680000
28597941405166726516864710559/208187937054666649077232435200
11482547005345338463969/85226428809510912000000
430374979754582929417781296799/3254501180508256899956736000000
3607856726470666022715979/27777728189842735104000000
200905120288615660597628246036915611/1573900804133279867023877210112000000
18497593486903125823791655511/147362699895661699242393600000
\end{lstlisting}


\bibliographystyle{siam}
\bibliography{refsnote} 



\end{document}
