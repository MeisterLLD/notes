\documentclass{notes}
\begin{document}
\entetenote{Toute fonction réelle convexe et dérivable est $\mathcal{C}^{1}$}{3}{}

\begin{theo*}
  Soit $I$ un intervalle de $\R$ et $f\colon I\to \R$ une fonction convexe et dérivable sur $I$. Alors $f$ est $\mathcal{C}^{1}$ sur $I$.
\end{theo*}

\begin{proof}
  On commence par un lemme qui explique que la seule chose qui peut rendre une fonction dérivable non $\mathcal{C}^{1}$ en un point, c'est une \emph{absence} de limite de la dérivée à gauche ou à droite de ce point.
  \begin{lemm*}
    Soit $g:I\to \R$ dérivable en $x_0$ mais pas $\mathcal{C}^{1}$ en $x_0$. Alors $g'$ n'admet pas de limite à gauche, ou n'admet pas de limite à droite, en $x_0$.
  \end{lemm*}
  \begin{proof}[Preuve du lemme]
    Le fait que $g'$ ne soit pas $\mathcal{C}^{1}$ en $x_0$ signifie que :
    \begin{itemize}
      \item ou bien $g'$ n'admet pas de limite à gauche ou n'admet pas de limite à droite en $x_0$,
      \item ou bien au moins l'une des limites à gauche ou à droite de $g'$ en $x_0$ n'est pas égale à $g'(x_0)$.
    \end{itemize}
    Par le théorème de Darboux, $g'$ vérifie le théorème des valeurs intermédiaires donc la deuxième alternative ne peut pas avoir lieu.
  \end{proof}

  La conclusion est maintenant immédiate, puisque $f'$ est croissante et admet donc une limite à gauche et à droite en tout $x_0\in I$.

\end{proof}


\bibliographystyle{siam}
\bibliography{refsnote} 



\end{document}
